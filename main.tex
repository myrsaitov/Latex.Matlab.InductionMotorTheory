\documentclass[12pt,a4paper]{scrartcl}
\usepackage{mathtext}
\usepackage[T2A]{fontenc}
\usepackage[utf8]{inputenc}
\usepackage[english,russian]{babel}
\usepackage{indentfirst}
\usepackage{misccorr}
\usepackage{graphicx}
\usepackage{amsmath, amsfonts}
\begin{document}
\begin{titlepage}
  \begin{center}
    \large
    МИНИСТЕРСТВО ОБРАЗОВАНИЯ И НАУКИ\\ РОССИЙСКОЙ ФЕДЕРАЦИИ\\
    \textbf{Федеральное агентство по образованию\\}
    \vspace{0.5cm}
    CЕВАСТОПОЛЬСКИЙ ГОСУДАРСТВЕННЫЙ УНИВЕРСИТЕТ\\
    \vspace{0.25cm}
    Морской институт\\
    Кафедра Судового электрооборудования\\
    \vfill
    Иванов Иван Иванович
    \vfill
    \textsc{Курсовая работа}\\[5mm]
    {\LARGE РАСЧЕТ  ЭКСПЛУАТАЦИОННЫХ ХАРАКТЕРИСТИК И ПАРАМЕТРОВ   АСИНХРОННЫХ ДВИГАТЕЛЕЙ\\}
  \bigskip
\end{center}
\vfill
\newlength{\ML}
\settowidth{\ML}{«\underline{\hspace{0.7cm}}» \underline{\hspace{2cm}}}
\hfill\begin{minipage}{0.5\textwidth}
  Руководитель курсовой работы\\
  \underline{\hspace{\ML}} В.\,С.~Высоцкий\\
  «\underline{\hspace{0.7cm}}» \underline{\hspace{2cm}} 2016 г.
\end{minipage}%
\bigskip
\bigskip
\begin{center}
  Севастополь, 2016 г.
\end{center}
\end{titlepage}
\section*{Введение}
\label{sec:intro}
\begin{enumerate}
 \item Формулировка проблемы
 \item Определение предмета исследования
 \item Определение цели исследования
 \item Постановка задач исследования
 \item Установка ограничений
 \item Определение необходимой информации
 \item Выявление объектов исследования
\end{enumerate}
\end{document} 
